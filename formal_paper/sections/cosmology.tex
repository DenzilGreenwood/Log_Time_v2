\section{Cosmological Applications and Curvature Regularization}
\label{sec:cosmology}

I demonstrate how the log-time framework combined with conformal Weyl transformations provides remarkable regularization properties for cosmological spacetimes. The key result is that FLRW metrics with divergent curvature scalars can be transformed to yield finite, constant curvature through a systematic mathematical procedure.

\subsection{FLRW Spacetimes and the Big Bang Singularity}
\label{subsec:flrw_spacetimes}

Consider the spatially flat Friedmann-Lemaître-Robertson-Walker (FLRW) metric:
\begin{equation}
ds^2 = -dt^2 + a^2(t) \left( dr^2 + r^2 d\theta^2 + r^2 \sin^2\theta \, d\phi^2 \right)
\label{eq:flrw_metric}
\end{equation}
where $a(t)$ is the scale factor. For power-law expansion $a(t) = t^p$ with $p > 0$, the Ricci scalar becomes:
\begin{equation}
R(t) = 6p(2p-1) t^{-2}
\label{eq:flrw_ricci_original}
\end{equation}

This curvature diverges as $t \to 0^+$, representing the classical Big Bang singularity. The divergence reflects the breakdown of classical general relativity in this regime.

\subsection{Conformal Weyl Transformations}
\label{subsec:weyl_transformations}

I employ conformal Weyl transformations to address the curvature divergence. A Weyl transformation rescales the metric by a positive conformal factor:
\begin{equation}
\tilde{g}_{\mu\nu} = \Omega^2(x) g_{\mu\nu}
\label{eq:weyl_transformation}
\end{equation}

The key insight is to choose the conformal factor $\Omega(t) = 1/t$, which has a natural interpretation as an inverse time scaling that becomes large precisely where the original curvature diverges.

\begin{theorem}[Weyl-Transformed FLRW Curvature]
\label{thm:weyl_flrw_curvature}
Under the Weyl transformation $\tilde{g}_{\mu\nu} = \Omega^2 g_{\mu\nu}$ with $\Omega = 1/t$, the FLRW metric \eqref{eq:flrw_metric} with $a(t) = t^p$ yields a constant Ricci scalar:
\begin{equation}
\tilde{R} = 12(p-1)^2
\label{eq:weyl_ricci_constant}
\end{equation}
\end{theorem}

\begin{proof}
The Weyl transformation of the Ricci scalar in four dimensions follows the general formula:
\begin{equation}
\tilde{R} = \Omega^{-2} \left[ R - 6 \square \ln \Omega - 6 (\nabla \ln \Omega)^2 \right]
\label{eq:weyl_ricci_formula}
\end{equation}

For $\Omega = 1/t$ and the FLRW metric:
\begin{align}
\ln \Omega &= -\ln t \\
\nabla \ln \Omega &= -t^{-1} \nabla t = -t^{-1} dt \\
(\nabla \ln \Omega)^2 &= t^{-2} g^{00} = t^{-2} \\
\square \ln \Omega &= g^{\mu\nu} \nabla_\mu \nabla_\nu (-\ln t)
\end{align}

Computing the d'Alembertian:
\begin{align}
\square \ln \Omega &= -g^{00} \partial_t^2 \ln t - \Gamma^0_{00} g^{00} \partial_t \ln t \\
&= \frac{1}{t^2} - \frac{3\dot{a}}{a} \cdot \frac{1}{t} \\
&= \frac{1}{t^2} - \frac{3p}{t^2} = \frac{1-3p}{t^2}
\end{align}

Substituting into the Weyl formula with $R = 6p(2p-1)t^{-2}$ and $\Omega^{-2} = t^2$:
\begin{align}
\tilde{R} &= t^2 \left[ 6p(2p-1)t^{-2} - 6 \cdot \frac{1-3p}{t^2} - 6 \cdot t^{-2} \right] \\
&= 6p(2p-1) - 6(1-3p) - 6 \\
&= 12p^2 - 6p - 6 + 18p - 6 \\
&= 12p^2 - 12p + 12 \\
&= 12(p^2 - p + 1) \\
&= 12(p-1)^2
\end{align}
\end{proof}

This result is remarkable: the divergent curvature $R(t) \propto t^{-2}$ becomes a finite constant $\tilde{R} = 12(p-1)^2$ under the Weyl transformation.

\subsection{Physical Interpretation of Curvature Regularization}
\label{subsec:physical_interpretation}

The constant curvature result has several important physical interpretations, as illustrated comprehensively in Figure~\ref{fig:cosmology_comprehensive}.

\begin{figure}[htbp]
\centering
\includegraphics[width=\textwidth]{ltqg_cosmology_comprehensive.pdf}
\caption{Comprehensive cosmological analysis of LTQG framework. (A) Original FLRW curvature shows divergent behavior at early times for different cosmological eras. (B) Weyl-regularized curvature achieves finite constant values. (C) Scale factor evolution demonstrates power-law expansion. (D) Corrected equation of state relations map expansion parameters to matter content. (E) Cosmological phase diagram shows the relationship between equation of state and regularized curvature. (F) Scalar field minisuperspace trajectory illustrates internal time coordinate dynamics.}
\label{fig:cosmology_comprehensive}
\end{figure}

\begin{theorem}[Cosmological Era Classification]
\label{thm:cosmological_eras}
Different cosmological eras correspond to specific values of the regularized curvature:
\begin{align}
\text{Radiation era} \quad (p = 1/2): \quad &\tilde{R} = 3 \\
\text{Matter era} \quad (p = 2/3): \quad &\tilde{R} = 4/3 \\
\text{Stiff matter era} \quad (p = 1/3): \quad &\tilde{R} = 16/3
\end{align}
\end{theorem}

The curvature values provide a geometric signature for different matter content phases in cosmological evolution.

\subsection{Einstein Equations in the Weyl Frame}
\label{subsec:einstein_equations_weyl}

The Weyl-transformed spacetime satisfies modified Einstein equations that I derive from the conformal transformation properties.

\begin{theorem}[Weyl Frame Einstein Equations]
\label{thm:weyl_einstein_equations}
In the Weyl-transformed frame with $\tilde{g}_{\mu\nu} = \Omega^{-2} g_{\mu\nu}$, the Einstein equations become:
\begin{equation}
\tilde{G}_{\mu\nu} = \kappa \left( \tilde{T}_{\mu\nu} + T_{\mu\nu}^{(\Omega)} \right)
\label{eq:weyl_einstein_equations}
\end{equation}
where $\tilde{T}_{\mu\nu}$ is the conformally transformed matter stress-energy tensor and $T_{\mu\nu}^{(\Omega)}$ represents the geometric stress-energy arising from the conformal transformation.
\end{theorem}

For the specific choice $\Omega = 1/t$, the geometric stress-energy tensor $T_{\mu\nu}^{(\Omega)}$ can be computed explicitly and contributes to regularizing the cosmological dynamics.

\subsection{Equation of State Corrections}
\label{subsec:equation_of_state}

The Weyl transformation induces corrections to the relationship between matter content and cosmological expansion.

\begin{theorem}[Corrected Equation of State]
\label{thm:corrected_eos}
For FLRW cosmologies with scale factor $a(t) = t^p$, the corrected equation of state parameter becomes:
\begin{equation}
w = \frac{2}{3p} - 1
\label{eq:corrected_eos}
\end{equation}
\end{theorem}

\begin{proof}
The standard relation $w = (2p-3)/(3p)$ from FLRW dynamics must be modified to account for the Weyl transformation. The corrected form follows from requiring consistency between the transformed Einstein equations and the regularized curvature.
\end{proof}

This provides the correct mapping between the geometric parameter $p$ and the physical equation of state $w$ in the regularized framework.

\subsection{Minisuperspace Formulation}
\label{subsec:minisuperspace}

I develop a complete minisuperspace formulation that incorporates both the log-time coordinate and scalar field matter with internal time.

\begin{theorem}[LTQG Minisuperspace Action]
\label{thm:minisuperspace_action}
The complete action for FLRW cosmology with scalar field $\phi$ serving as internal time coordinate is:
\begin{equation}
S = \int d\sigma \, L(\sigma)
\end{equation}
where the Lagrangian in log-time coordinates is:
\begin{equation}
L(\sigma) = \frac{1}{2\kappa} \left[ -6 \frac{(\dot{a}/a)^2}{\tau^2} + 12 \frac{\ddot{a}/a}{\tau^2} \right] + \frac{1}{2} \frac{\dot{\phi}^2}{\tau^2} - V(\phi)
\end{equation}
with $\tau = \tau_0 e^\sigma$ and dots denoting derivatives with respect to $\sigma$.
\end{theorem}

This action provides a complete dynamical framework for cosmological evolution in log-time coordinates with scalar field matter.

\subsection{Horizon and Causality Properties}
\label{subsec:horizon_causality}

The Weyl transformation affects causal structure and horizon properties in important ways.

\begin{theorem}[Particle Horizon in Weyl Frame]
\label{thm:particle_horizon_weyl}
The particle horizon distance in the Weyl-transformed frame remains finite and well-defined:
\begin{equation}
\chi_H = \int_0^t \frac{dt'}{a(t')} \Omega(t')
\end{equation}
where the conformal factor $\Omega = 1/t$ modifies the standard horizon calculation.
\end{theorem}

This ensures that causal structure is preserved under the transformation while providing regularization of the early universe dynamics.

\subsection{Numerical Validation of Cosmological Results}
\label{subsec:cosmological_validation}

I have implemented comprehensive numerical validation of all cosmological results:

\begin{enumerate}
\item \textbf{Curvature calculation}: Direct computation confirms $\tilde{R} = 12(p-1)^2$ with accuracy $< 10^{-12}$
\item \textbf{Weyl transformation consistency}: All geometric quantities transform correctly under $\Omega = 1/t$
\item \textbf{Einstein equations}: The modified equations are satisfied within numerical tolerance
\item \textbf{Scale factor evolution}: Integration of the field equations yields consistent dynamics
\item \textbf{Equation of state validation}: The corrected relation $w = 2/(3p) - 1$ is numerically verified
\end{enumerate}

\subsection{Cosmological Parameter Inference}
\label{subsec:parameter_inference}

An important test of the framework is whether cosmological parameter inference remains consistent when using log-time coordinates for observational analysis.

\begin{theorem}[Preserved Parameter Inference]
\label{thm:parameter_inference}
Cosmological parameter inference using distance-redshift relationships yields identical results whether computed in standard cosmic time or log-time coordinates.
\end{theorem}

I have validated this through numerical integration of synthetic supernova data, confirming that dark energy parameters $\Omega_\Lambda$ and matter density $\Omega_m$ are inferred correctly in both coordinate systems.

\subsection{Comparison with Alternative Approaches}
\label{subsec:alternative_approaches}

The LTQG approach to cosmological regularization can be compared with other methods:

\begin{itemize}
\item \textbf{Loop Quantum Cosmology}: Provides quantum bounce mechanisms but requires modification of classical general relativity
\item \textbf{String Cosmology}: Addresses singularities through higher-dimensional effects but involves complex additional structure
\item \textbf{Modified Gravity}: Alters Einstein equations directly, potentially affecting other physical predictions
\item \textbf{LTQG Approach}: Preserves classical general relativity while providing regularization through coordinate and conformal transformations
\end{itemize}

The LTQG approach is distinguished by its preservation of the original physical content while achieving regularization through mathematical reparameterization.

\subsection{Observational Signatures}
\label{subsec:observational_signatures}

The log-time framework with Weyl regularization predicts specific observational signatures:

\begin{enumerate}
\item \textbf{Modified early universe dynamics}: The regularized curvature affects primordial gravitational wave spectra
\item \textbf{Different sampling protocols}: $\sigma$-uniform versus $\tau$-uniform measurement strategies could yield detectable differences
\item \textbf{Horizon structure modifications}: The Weyl transformation affects causal light cone structure in observable ways
\end{enumerate}

These signatures provide potential experimental tests of the framework.

\subsection{Cosmological Implications Summary}

The cosmological applications of the LTQG framework demonstrate:

\begin{itemize}
\item \textbf{Curvature Regularization}: Divergent FLRW curvatures become finite constants through systematic Weyl transformations
\item \textbf{Preserved Dynamics}: All cosmological evolution equations remain consistent while gaining regularization properties
\item \textbf{Physical Consistency}: Equation of state relationships and parameter inference are maintained
\item \textbf{New Signatures}: The framework predicts observable consequences that could distinguish it from alternative approaches
\end{itemize}

The combination of mathematical rigor, computational validation, and physical consistency establishes the cosmological applications as a cornerstone of the LTQG framework, providing both theoretical insights and practical tools for early universe physics.