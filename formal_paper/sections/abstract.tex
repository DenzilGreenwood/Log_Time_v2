\begin{abstract}
I present a mathematically rigorous framework that bridges General Relativity and Quantum Mechanics through temporal reparameterization, which I term Log-Time Quantum Gravity (LTQG). The central insight I develop is that the logarithmic time coordinate $\sigma := \log(\tau/\tau_0)$ (defined for $\tau > 0$) converts General Relativity's multiplicative time-dilation factors into additive shifts, naturally aligning with Quantum Mechanics' additive phase evolution.

I establish four fundamental mathematical results: First, I prove the exact invertibility and unitary equivalence of quantum evolution between proper time $\tau$ and log-time $\sigma$ coordinates, preserving all physical predictions (proved in §\ref{sec:quantum_mechanics}) while enabling asymptotic silence as $\sigma \to -\infty$. Second, I demonstrate that pairing this temporal reparameterization with conformal Weyl transformations—a separate geometric analysis from the re-clocking—regularizes curvature singularities in FLRW cosmologies, yielding finite constant curvature $\tilde{R} = 12(p-1)^2$ for scale factors $a(t) = t^p$. Third, I show that quantum field theory mode evolution maintains Wronskian conservation and Bogoliubov unitarity across coordinate systems with sub-$10^{-6}$ numerical precision. Fourth, I identify operational distinctions between $\sigma$-uniform and $\tau$-uniform measurement protocols: $\sigma$-uniform sampling provides exponentially increasing resolution toward early times, creating measurably different data collection patterns compared to uniform proper time intervals.

My comprehensive computational validation suite, implementing these theoretical results through symbolic computation and high-precision numerical analysis, confirms all mathematical claims to machine precision. The framework preserves the complete physical content of both General Relativity and Quantum Mechanics while providing new tools for studying early universe cosmology and quantum gravity phenomenology. Concrete demonstrations focus on FLRW/minisuperspace applications, with black hole physics representing prospective rather than demonstrated extensions.

This work represents a reparameterization approach rather than a modification of existing theories, offering a mathematically consistent bridge between the multiplicative temporal structure of spacetime geometry and the additive temporal evolution of quantum systems. However, the framework faces two fundamental conceptual limitations that prevent it from being a complete quantum gravity solution: (1) curvature regularization in the Weyl frame does not automatically resolve geodesic incompleteness in the original frame, and frame-dependence requires external matter coupling prescriptions to determine physical interpretation; (2) the approach sidesteps rather than resolves the canonical Problem of Time, as diffeomorphism invariance cannot be preserved while maintaining well-defined evolution. These limitations establish LTQG as a powerful computational and theoretical tool rather than a fundamental theory of quantum gravity. The computational implementation provides reproducible verification of all theoretical claims and serves as a foundation for future research applications in cosmology and early universe physics.
\end{abstract}