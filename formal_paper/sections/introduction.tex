\section{Introduction}
\label{sec:introduction}

The unification of General Relativity (GR) and Quantum Mechanics (QM) remains one of the most profound challenges in theoretical physics. While both theories have achieved remarkable empirical success within their respective domains, their fundamental mathematical structures exhibit a deep tension when considered together. I identify this tension as primarily temporal in nature: General Relativity treats time transformations as multiplicative operations through coordinate changes and gravitational redshift, while Quantum Mechanics evolves quantum states through additive phase accumulation with respect to an external time parameter.

\subsection{The Multiplicative-Additive Temporal Clash}

In General Relativity, the proper time interval $d\tau$ between events transforms under coordinate changes and in gravitational fields according to multiplicative factors. For a coordinate transformation $t \to t'$, local clock rates scale as $d\tau' = \gamma(t) d\tau$ where $\gamma(t)$ is a position and time-dependent factor. In static spacetimes, this manifests as $d\tau = \sqrt{-g_{tt}} dt$ where the metric component provides the multiplicative redshift factor. Gravitational redshift similarly manifests as multiplicative relationships between proper times measured by different observers.

Quantum Mechanics, conversely, evolves quantum states according to the Schrödinger equation:
\begin{equation}
i\hbar \frac{\partial \psi}{\partial t} = H(t) \psi
\end{equation}
where phases accumulate additively: $\psi(t) = \exp\left(-\frac{i}{\hbar}\int_0^t H(t') dt'\right) \psi(0)$. This additive structure is fundamental to quantum superposition, interference, and the linearity of quantum evolution.

The mathematical incompatibility becomes acute near classical singularities where gravitational fields diverge. In such regions, multiplicative factors in General Relativity become infinite while Quantum Mechanics requires well-defined, finite phase evolution. Traditional approaches to quantum gravity attempt to resolve this tension by modifying one or both theories, often at the cost of mathematical complexity and loss of experimental connection.

\subsection{The Logarithmic Resolution}

I propose a different approach: rather than modifying the physical theories themselves, I reparameterize their temporal coordinates to achieve mathematical compatibility. The key insight is that the logarithmic function converts multiplication into addition: $\log(ab) = \log a + \log b$. This mathematical property suggests that a logarithmic time coordinate might bridge the multiplicative-additive divide.

I define the log-time coordinate as:
\begin{equation}
\sigma = \log\left(\frac{\tau}{\tau_0}\right)
\end{equation}
where $\tau > 0$ is proper time and $\tau_0 > 0$ is a positive reference scale. Throughout we assume $\tau > 0$ along each timelike worldline, so $\sigma = \log(\tau/\tau_0)$ is a $C^1$ bijection $\mathbb{R}^+ \to \mathbb{R}$. All results in Sections 2–3 are therefore reparameterizations of the same dynamics, not modifications of the Hamiltonian theory.

Under this transformation, any multiplicative redshift $\tau' = \gamma \tau$ becomes an additive shift $\sigma' = \sigma + \log \gamma$. Crucially, this preserves causal ordering since $d\sigma/d\tau = 1/\tau > 0$ for all $\tau > 0$.

This reparameterization converts General Relativity's multiplicative time structure into an additive form compatible with Quantum Mechanics' phase evolution, while preserving all physical predictions of both theories. Note that this $\sigma$-clock transformation leaves physics invariant; the Weyl rescaling explored in §\ref{sec:cosmology} is a distinct geometric analysis.

\subsection{Mathematical Framework Overview}

The Log-Time Quantum Gravity (LTQG) framework I develop consists of four interconnected mathematical components:

\paragraph{Temporal Reparameterization} The fundamental coordinate transformation $\sigma = \log(\tau/\tau_0)$ with its inverse $\tau = \tau_0 e^\sigma$ provides an exact, invertible mapping between proper time and log-time coordinates. I prove that this transformation preserves all differential relationships while converting multiplicative operations into additive ones.

\paragraph{Quantum Evolution Equivalence} I establish unitary equivalence between quantum evolution in $\tau$ and $\sigma$ coordinates through the transformed Schrödinger equation:
\begin{equation}
i\hbar \frac{\partial \psi}{\partial \sigma} = \tau_0 e^\sigma H(\tau_0 e^\sigma) \psi = K(\sigma) \psi
\end{equation}
The effective generator $K(\sigma)$ inherits Hermiticity from $H(\tau)$ while exhibiting asymptotic silence as $\sigma \to -\infty$.

\paragraph{Geometric Regularization} I demonstrate that pairing the log-time coordinate with conformal Weyl transformations $\tilde{g}_{\mu\nu} = \Omega^2 g_{\mu\nu}$ regularizes curvature divergences in cosmological spacetimes. For FLRW metrics with $\Omega = 1/t$, I obtain finite constant curvature $\tilde{R} = 12(p-1)^2$ replacing the divergent behavior $R \propto t^{-2}$.

\paragraph{Operational Consequences} The framework predicts measurable distinctions between $\sigma$-uniform and $\tau$-uniform measurement protocols. These arise because "uniform in $\sigma$" corresponds to exponentially spaced intervals in $\tau$, creating different sampling patterns for experimental observations.

\subsection{Computational Validation}

I have implemented the complete theoretical framework in a comprehensive Python validation suite that verifies all mathematical claims through both symbolic computation and high-precision numerical analysis. The validation covers:
\begin{itemize}
\item Round-trip coordinate transformation accuracy to machine precision
\item Quantum unitary equivalence for both constant and time-dependent Hamiltonians
\item Cosmological curvature regularization across different matter content
\item Quantum field theory mode evolution with Wronskian and Bogoliubov conservation
\item Complete geometric analysis including curvature tensors and invariants
\end{itemize}

This computational verification ensures that every theoretical claim I make has been rigorously tested and confirmed within numerical tolerances appropriate for each physical domain.

\subsection{Scope and Limitations}

The LTQG framework I present is a reparameterization approach, not a new physical theory. It preserves the complete mathematical and physical content of both General Relativity and Quantum Mechanics while providing new computational and conceptual tools for their unified treatment. The framework does not resolve all conceptual issues in quantum gravity—such as the measurement problem or the nature of spacetime at the Planck scale—but it does provide a mathematically consistent foundation for addressing temporal aspects of the unification challenge.

I acknowledge \textbf{two fundamental conceptual limitations} that prevent LTQG from being a complete solution to quantum gravity: \textbf{(1) Ambiguity of singularity resolution}—while curvature regularization is achieved through Weyl transformations, this does not automatically resolve geodesic incompleteness in the original spacetime frame, and the frame-dependence problem requires matter coupling prescriptions to determine physical interpretation; \textbf{(2) The Problem of Time}—the reparameterization approach sidesteps rather than fundamentally resolves the ``frozen formalism'' arising from diffeomorphism invariance in canonical quantum gravity, representing a deparameterization technique that works in minisuperspace but faces conceptual challenges in full field theory.

Additional limitations include: the geometric analysis beyond scalar curvature requires completion of higher-order invariants, the field theory implementation needs extension to interacting theories and renormalization, and the experimental accessibility of $\sigma$-uniform protocols requires detailed investigation. These limitations establish LTQG as a powerful computational and theoretical tool rather than a fundamental theory of quantum gravity.

\subsection{Organization of This Work}

The remainder of this paper is organized as follows: Section~\ref{sec:mathematical_framework} establishes the rigorous mathematical foundations of the log-time transformation and its properties. Section~\ref{sec:quantum_mechanics} proves unitary equivalence and asymptotic silence in quantum evolution. Section~\ref{sec:cosmology} demonstrates curvature regularization in cosmological spacetimes through Weyl transformations. Section~\ref{sec:qft} extends the framework to quantum field theory on curved backgrounds. Section~\ref{sec:computational_validation} documents the comprehensive validation suite and its results. Section~\ref{sec:results_discussion} synthesizes the key findings and their implications. Section~\ref{sec:conclusion} summarizes the contributions and outlines future directions.

This systematic development ensures that each component of the framework builds rigorously upon previous results while maintaining clear connections to both the underlying physics and the computational implementation that validates all theoretical claims.