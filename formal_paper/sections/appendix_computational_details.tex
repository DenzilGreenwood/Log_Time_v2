\section{Computational Implementation Details}
\label{app:computational_details}

This appendix provides detailed information about the computational implementation of the LTQG framework, including algorithms, numerical methods, software architecture, and reproducibility protocols.

\subsection{Software Architecture Overview}
\label{app:software_architecture}

The LTQG computational framework is implemented as a modular Python package with the following structure:

\begin{verbatim}
ltqg/
|-- ltqg_core.py              # Core mathematical foundations
|-- ltqg_quantum.py           # Quantum mechanical applications
|-- ltqg_cosmology.py         # Cosmological dynamics
|-- ltqg_qft.py              # Quantum field theory
|-- ltqg_curvature.py        # Curvature analysis
|-- ltqg_variational.py      # Variational mechanics
|-- ltqg_main.py             # Validation orchestration
|-- ltqg_validation_extended.py  # Extended validation suite
`-- webgl/                   # Interactive visualizations
    |-- ltqg_black_hole_webgl.html
    |-- ltqg_bigbang_funnel.html
    `-- serve_webgl.py
\end{verbatim}

Each module follows a consistent design pattern:
\begin{enumerate}
\item Import statements and configuration
\item Utility functions and mathematical tools
\item Core implementation classes and functions
\item Validation and testing functions
\item Demonstration and example code
\end{enumerate}

\subsection{Numerical Integration Methods}
\label{app:numerical_integration}

The framework employs several numerical integration schemes optimized for different types of differential equations:

\subsubsection{Adaptive Runge-Kutta Methods}

For general ODEs, we use adaptive RK45 integration with the following parameters:
\begin{itemize}
\item Relative tolerance: $10^{-10}$
\item Absolute tolerance: $10^{-12}$
\item Maximum step size: $\Delta\sigma_{\max} = 0.1$
\item Minimum step size: $\Delta\sigma_{\min} = 10^{-8}$
\end{itemize}

The implementation uses SciPy's \texttt{solve\_ivp} with the \texttt{'RK45'} method:

\begin{verbatim}
from scipy.integrate import solve_ivp

def integrate_mode_equation(sigma_span, initial_conditions, params):
    def mode_ode(sigma, y):
        u, w = y
        dudt = w
        dwdt = -(1-3*params['p'])*w - params['omega_sq'](sigma)*u
        return [dudt, dwdt]
    
    sol = solve_ivp(mode_ode, sigma_span, initial_conditions,
                    method='RK45', rtol=1e-10, atol=1e-12,
                    max_step=0.1, dense_output=True)
    return sol
\end{verbatim}

\subsubsection{Symplectic Integration for Hamiltonian Systems}

For quantum evolution with Hamiltonian structure, we implement symplectic integrators to preserve the symplectic structure of phase space:

\begin{verbatim}
def symplectic_evolution(H, psi0, sigma_span, n_steps):
    """Symplectic integration for quantum evolution."""
    sigma_i, sigma_f = sigma_span
    dt = (sigma_f - sigma_i) / n_steps
    
    psi = psi0.copy()
    for i in range(n_steps):
        sigma = sigma_i + i * dt
        
        # Symplectic step: exp(-iH*dt/2) exp(-iT*dt) exp(-iH*dt/2)
        # where H = T + V, assuming separable Hamiltonian
        psi = expm(-1j * H(sigma) * dt / 2) @ psi
        psi = expm(-1j * H(sigma + dt/2) * dt / 2) @ psi
        psi = expm(-1j * H(sigma + dt) * dt / 2) @ psi
    
    return psi
\end{verbatim}

\subsubsection{Stiff ODE Methods}

For equations with rapidly varying coefficients near singular points, we use implicit methods:

\begin{verbatim}
def solve_stiff_system(sigma_span, y0, jacobian_func):
    """Solve stiff ODE system using backward differentiation."""
    sol = solve_ivp(ode_func, sigma_span, y0, method='BDF',
                    jac=jacobian_func, rtol=1e-8, atol=1e-10)
    return sol
\end{verbatim}

\subsection{Symbolic Computation Implementation}
\label{app:symbolic_computation}

The framework extensively uses SymPy for exact symbolic computation and verification:

\subsubsection{Curvature Tensor Calculation}

\begin{verbatim}
import sympy as sp
from sympy import symbols, Matrix, simplify, diff

def compute_flrw_curvature_symbolic():
    """Symbolic computation of FLRW curvature."""
    t, r, theta, phi, p = symbols('t r theta phi p', real=True, positive=True)
    
    # Define metric components
    g = Matrix([
        [-1, 0, 0, 0],
        [0, t**(2*p), 0, 0],
        [0, 0, t**(2*p) * r**2, 0],
        [0, 0, 0, t**(2*p) * r**2 * sp.sin(theta)**2]
    ])
    
    # Compute Christoffel symbols
    christoffel = compute_christoffel_symbols(g, [t, r, theta, phi])
    
    # Compute Riemann tensor
    riemann = compute_riemann_tensor(christoffel, [t, r, theta, phi])
    
    # Compute Ricci tensor and scalar
    ricci_tensor = contract_riemann_to_ricci(riemann, g)
    ricci_scalar = contract_ricci_to_scalar(ricci_tensor, g)
    
    return simplify(ricci_scalar)

def compute_christoffel_symbols(metric, coords):
    """Compute Christoffel symbols symbolically."""
    n = len(coords)
    christoffel = [[[0 for k in range(n)] for j in range(n)] for i in range(n)]
    
    g_inv = metric.inv()
    
    for i in range(n):
        for j in range(n):
            for k in range(n):
                christoffel[i][j][k] = sum(
                    g_inv[i, l] * (
                        diff(metric[l, j], coords[k]) +
                        diff(metric[l, k], coords[j]) -
                        diff(metric[j, k], coords[l])
                    ) / 2
                    for l in range(n)
                )
    
    return christoffel
\end{verbatim}

\subsubsection{Weyl Transformation Verification}

\begin{verbatim}
def verify_weyl_transformation_symbolic():
    """Symbolically verify Weyl transformation results."""
    t, p = symbols('t p', real=True, positive=True)
    
    # Original Ricci scalar
    R_original = 6*p*(2*p-1)/t**2
    
    # Weyl factor
    Omega = 1/t
    
    # Compute transformed curvature
    ln_Omega = sp.log(Omega)
    grad_ln_Omega_sq = (diff(ln_Omega, t))**2
    dalembertian_ln_Omega = diff(diff(ln_Omega, t), t)  # Simplified for FLRW
    
    R_weyl = Omega**(-2) * (R_original - 6*dalembertian_ln_Omega - 6*grad_ln_Omega_sq)
    
    R_weyl_simplified = simplify(R_weyl)
    
    # Verify result
    expected = 12*(p-1)**2
    assert simplify(R_weyl_simplified - expected) == 0
    
    return R_weyl_simplified
\end{verbatim}

\subsection{High-Precision Arithmetic}
\label{app:high_precision}

For calculations requiring extreme precision, the framework supports arbitrary precision arithmetic:

\begin{verbatim}
from decimal import Decimal, getcontext
import mpmath as mp

def high_precision_calculation(precision_digits=50):
    """Perform calculations with arbitrary precision."""
    # Set precision
    getcontext().prec = precision_digits
    mp.mp.dps = precision_digits
    
    # Example: high-precision log-time transformation
    tau0 = mp.mpf('1.0')
    tau = mp.mpf('1e-20')
    
    sigma = mp.log(tau / tau0)
    tau_reconstructed = tau0 * mp.exp(sigma)
    
    error = abs(tau - tau_reconstructed)
    
    return float(error)

def validate_round_trip_high_precision():
    """Validate round-trip accuracy with high precision."""
    errors = []
    for precision in [50, 100, 200]:
        error = high_precision_calculation(precision)
        errors.append(error)
        print(f"Precision {precision}: error = {error}")
    
    return errors
\end{verbatim}

\subsection{Parallel Computing Implementation}
\label{app:parallel_computing}

The validation suite supports parallel execution for independent test cases:

\begin{verbatim}
import multiprocessing as mp
from concurrent.futures import ProcessPoolExecutor, as_completed

def parallel_validation_suite():
    """Run validation tests in parallel."""
    test_functions = [
        run_core_validation_suite,
        run_quantum_evolution_validation,
        run_cosmology_validation,
        run_qft_validation,
        run_curvature_analysis_validation,
        run_variational_mechanics_validation
    ]
    
    results = {}
    
    with ProcessPoolExecutor(max_workers=mp.cpu_count()) as executor:
        future_to_test = {
            executor.submit(test_func): test_func.__name__
            for test_func in test_functions
        }
        
        for future in as_completed(future_to_test):
            test_name = future_to_test[future]
            try:
                result = future.result()
                results[test_name] = {'status': 'PASS', 'result': result}
            except Exception as exc:
                results[test_name] = {'status': 'FAIL', 'error': str(exc)}
    
    return results
\end{verbatim}

\subsection{Error Analysis and Convergence Testing}
\label{app:error_analysis}

The framework includes comprehensive error analysis tools:

\subsubsection{Richardson Extrapolation}

\begin{verbatim}
def richardson_extrapolation(f, h_values, order=2):
    """Perform Richardson extrapolation to estimate limiting value."""
    if len(h_values) < 2:
        raise ValueError("Need at least 2 h values for extrapolation")
    
    # Compute function values
    f_values = [f(h) for h in h_values]
    
    # Richardson extrapolation formula
    h1, h2 = h_values[-2], h_values[-1]
    f1, f2 = f_values[-2], f_values[-1]
    
    ratio = h1 / h2
    extrapolated = f2 + (f2 - f1) / (ratio**order - 1)
    
    return extrapolated

def convergence_test(integration_function, h_sequence):
    """Test convergence of numerical integration."""
    errors = []
    
    for h in h_sequence:
        result = integration_function(step_size=h)
        # Compare with reference solution or higher-order method
        reference = reference_solution()
        error = abs(result - reference)
        errors.append(error)
    
    # Estimate convergence order
    log_errors = [np.log(err) for err in errors]
    log_h = [np.log(h) for h in h_sequence]
    
    # Linear fit to log-log data
    slope, intercept = np.polyfit(log_h, log_errors, 1)
    
    return slope, errors  # slope gives convergence order
\end{verbatim}

\subsubsection{Adaptive Error Control}

\begin{verbatim}
def adaptive_integration_with_error_control(ode_func, span, y0, tol):
    """Adaptive integration with automatic error control."""
    sigma_start, sigma_end = span
    h = 0.01  # Initial step size
    
    sigma = sigma_start
    y = y0.copy()
    
    results = [(sigma, y.copy())]
    
    while sigma < sigma_end:
        # Take step with current h
        y1 = rk4_step(ode_func, sigma, y, h)
        
        # Take two steps with h/2
        y_half = rk4_step(ode_func, sigma, y, h/2)
        y2 = rk4_step(ode_func, sigma + h/2, y_half, h/2)
        
        # Estimate error
        error = np.linalg.norm(y2 - y1)
        
        if error < tol:
            # Accept step
            sigma += h
            y = y2
            results.append((sigma, y.copy()))
            
            # Increase step size if error is very small
            if error < tol / 10:
                h = min(h * 1.5, 0.1)
        else:
            # Reject step and reduce step size
            h = h * 0.5
            if h < 1e-8:
                raise RuntimeError("Step size too small - integration failed")
    
    return results
\end{verbatim}

\subsection{Reproducibility Protocols}
\label{app:reproducibility}

The framework implements strict reproducibility protocols:

\subsubsection{Deterministic Random Number Generation}

\begin{verbatim}
import numpy as np
import random

def set_reproducible_environment():
    """Set all random seeds for reproducible results."""
    # Set numpy random seed
    np.random.seed(12345)
    
    # Set Python random seed
    random.seed(12345)
    
    # Set environment variables for deterministic behavior
    import os
    os.environ['PYTHONHASHSEED'] = '0'
    
    # For multiprocessing reproducibility
    import multiprocessing as mp
    mp.set_start_method('spawn', force=True)

def reproducibility_test():
    """Test that computations are reproducible."""
    results1 = []
    results2 = []
    
    # Run calculations twice with same seeds
    for _ in range(2):
        set_reproducible_environment()
        
        # Run a sample calculation
        result = run_quantum_evolution_validation()
        if len(results1) == 0:
            results1 = result
        else:
            results2 = result
    
    # Verify identical results
    assert np.allclose(results1, results2, rtol=1e-15, atol=1e-15)
    
    return True
\end{verbatim}

\subsubsection{Environment Documentation}

\begin{verbatim}
def document_computational_environment():
    """Document the computational environment for reproducibility."""
    import sys
    import platform
    import numpy
    import scipy
    import sympy
    
    env_info = {
        'python_version': sys.version,
        'platform': platform.platform(),
        'processor': platform.processor(),
        'numpy_version': numpy.__version__,
        'scipy_version': scipy.__version__,
        'sympy_version': sympy.__version__,
        'python_executable': sys.executable,
        'python_path': sys.path
    }
    
    # Save to file
    import json
    with open('computational_environment.json', 'w') as f:
        json.dump(env_info, f, indent=2)
    
    return env_info
\end{verbatim}

\subsection{Performance Optimization}
\label{app:performance_optimization}

The framework includes several optimization strategies:

\subsubsection{Vectorized Operations}

\begin{verbatim}
def vectorized_log_time_transformation(tau_array, tau0):
    """Vectorized log-time transformation for efficiency."""
    # Input validation
    tau_array = np.asarray(tau_array)
    if np.any(tau_array <= 0):
        raise ValueError("All tau values must be positive")
    
    # Vectorized computation
    sigma_array = np.log(tau_array / tau0)
    
    return sigma_array

def vectorized_mode_evolution(k_values, sigma_span, params):
    """Vectorized mode evolution for multiple k values."""
    n_modes = len(k_values)
    n_sigma = len(sigma_span)
    
    # Preallocate arrays
    u_results = np.zeros((n_modes, n_sigma), dtype=complex)
    w_results = np.zeros((n_modes, n_sigma), dtype=complex)
    
    # Vectorized evolution
    for i, k in enumerate(k_values):
        mode_params = params.copy()
        mode_params['k'] = k
        
        sol = integrate_mode_equation(sigma_span, [1.0, 0.0], mode_params)
        u_results[i, :] = sol.y[0]
        w_results[i, :] = sol.y[1]
    
    return u_results, w_results
\end{verbatim}

\subsubsection{Memory-Efficient Algorithms}

\begin{verbatim}
def memory_efficient_long_evolution(ode_func, span, y0, chunk_size=1000):
    """Memory-efficient evolution for long time series."""
    sigma_start, sigma_end = span
    
    # Generator function for streaming results
    def evolution_generator():
        current_sigma = sigma_start
        current_y = y0.copy()
        
        while current_sigma < sigma_end:
            # Evolve for one chunk
            chunk_end = min(current_sigma + chunk_size * 0.01, sigma_end)
            
            sol = solve_ivp(ode_func, [current_sigma, chunk_end], current_y,
                           dense_output=False, rtol=1e-10, atol=1e-12)
            
            # Yield results for this chunk
            for i in range(len(sol.t)):
                yield sol.t[i], sol.y[:, i]
            
            # Update for next chunk
            current_sigma = chunk_end
            current_y = sol.y[:, -1]
    
    return evolution_generator()
\end{verbatim}

\subsection{Testing and Continuous Integration}
\label{app:testing_ci}

The framework includes comprehensive testing infrastructure:

\begin{verbatim}
import unittest
import pytest

class TestLTQGCore(unittest.TestCase):
    """Test cases for core LTQG functionality."""
    
    def setUp(self):
        """Set up test fixtures."""
        set_reproducible_environment()
        self.tau0 = 1.0
        self.test_precision = 1e-12
    
    def test_round_trip_accuracy(self):
        """Test round-trip transformation accuracy."""
        tau_values = np.logspace(-20, 20, 100)
        
        for tau in tau_values:
            sigma = np.log(tau / self.tau0)
            tau_reconstructed = self.tau0 * np.exp(sigma)
            
            relative_error = abs(tau - tau_reconstructed) / tau
            self.assertLess(relative_error, self.test_precision)
    
    def test_asymptotic_silence(self):
        """Test asymptotic silence property."""
        def test_hamiltonian(tau):
            return np.array([[1/tau**0.5, 0], [0, 2/tau**0.5]])
        
        sigma_values = np.linspace(-10, 0, 100)
        
        for sigma in sigma_values:
            tau = self.tau0 * np.exp(sigma)
            K_sigma = tau * test_hamiltonian(tau)
            
            # Verify decreasing behavior
            if sigma < -1:
                self.assertLess(np.linalg.norm(K_sigma), 10)

if __name__ == '__main__':
    unittest.main()
\end{verbatim}

This comprehensive computational implementation ensures that all theoretical results are verified through rigorous numerical analysis while maintaining reproducibility and extensibility for future research applications.