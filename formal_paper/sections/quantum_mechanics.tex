\section{Quantum Mechanics in Log-Time Coordinates}
\label{sec:quantum_mechanics}

I establish the quantum mechanical foundations of the LTQG framework by deriving the log-time Schrödinger equation, proving unitary equivalence between $\tau$ and $\sigma$ evolution, and demonstrating asymptotic silence. These results show that the temporal reparameterization preserves all quantum mechanical predictions while providing new mathematical structure.

\subsection{The Log-Time Schrödinger Equation}
\label{subsec:sigma_schrodinger}

The transformation of the Schrödinger equation from proper time to log-time coordinates is achieved through direct application of the chain rule established in Section~\ref{subsec:differential_calculus}.

\begin{theorem}[Log-Time Schrödinger Equation]
\label{thm:sigma_schrodinger}
Consider the standard Schrödinger equation in proper time:
\begin{equation}
i\hbar \frac{\partial \psi}{\partial \tau} = H(\tau) \psi
\label{eq:tau_schrodinger}
\end{equation}
Under the log-time transformation $\sigma = \log(\tau/\tau_0)$, this becomes:
\begin{equation}
i\hbar \frac{\partial \psi}{\partial \sigma} = K(\sigma) \psi
\label{eq:sigma_schrodinger}
\end{equation}
where the effective generator is:
\begin{equation}
K(\sigma) = \tau_0 e^\sigma H(\tau_0 e^\sigma)
\label{eq:effective_generator}
\end{equation}
\end{theorem}

\begin{proof}
Applying the chain rule from Theorem~\ref{thm:chain_rule}:
\begin{align}
i\hbar \frac{\partial \psi}{\partial \tau} &= i\hbar \frac{1}{\tau} \frac{\partial \psi}{\partial \sigma} \\
&= i\hbar \frac{1}{\tau_0 e^\sigma} \frac{\partial \psi}{\partial \sigma}
\end{align}
Equating with the original Schrödinger equation:
\begin{align}
i\hbar \frac{1}{\tau_0 e^\sigma} \frac{\partial \psi}{\partial \sigma} &= H(\tau) \psi \\
i\hbar \frac{\partial \psi}{\partial \sigma} &= \tau_0 e^\sigma H(\tau_0 e^\sigma) \psi
\end{align}
which establishes equation \eqref{eq:sigma_schrodinger} with the effective generator \eqref{eq:effective_generator}.
\end{proof}

\begin{corollary}[Hermiticity Preservation]
If $H(\tau)$ is Hermitian for all $\tau$, then $K(\sigma)$ is Hermitian for all $\sigma$.
\end{corollary}

\begin{proof}
Since $\tau_0 e^\sigma$ is real and positive, $K(\sigma) = \tau_0 e^\sigma H(\tau_0 e^\sigma)$ inherits Hermiticity from $H(\tau)$:
\begin{equation}
K(\sigma)^\dagger = (\tau_0 e^\sigma)^* [H(\tau_0 e^\sigma)]^\dagger = \tau_0 e^\sigma H(\tau_0 e^\sigma) = K(\sigma)
\end{equation}
\end{proof}

This preservation of Hermiticity ensures that quantum evolution remains unitary in log-time coordinates.

\emph{Important Note}: We emphasize $K(\sigma)$ is the generator in $\sigma$ induced by the Jacobian $d\tau = \tau_0 e^\sigma d\sigma$; it is not a new Hamiltonian but a bookkeeping re-expression of $H$ under the clock change. The physics remains unchanged; only the temporal parameterization differs.

\subsection{Unitary Equivalence of Evolution Operators}
\label{subsec:unitary_equivalence}

The central physical requirement is that quantum evolution in $\tau$ and $\sigma$ coordinates must yield identical predictions for all observables. I establish this through rigorous analysis of time-ordered evolution operators.

\begin{theorem}[Unitary Equivalence for Constant Hamiltonians]
\label{thm:unitary_equiv_constant}
For time-independent Hamiltonians $H(\tau) = H_0$, the evolution operators satisfy:
\begin{equation}
U_\tau(t_f, t_i) = U_\sigma(\sigma_f, \sigma_i)
\end{equation}
where:
\begin{align}
U_\tau(t_f, t_i) &= \exp\left(-\frac{i}{\hbar} H_0 (t_f - t_i)\right) \\
U_\sigma(\sigma_f, \sigma_i) &= \exp\left(-\frac{i}{\hbar} \int_{\sigma_i}^{\sigma_f} K(\sigma') d\sigma'\right)
\end{align}
\end{theorem}

\begin{proof}
For constant $H_0$, the effective generator becomes $K(\sigma) = \tau_0 e^\sigma H_0$. The integral evaluates to:
\begin{align}
\int_{\sigma_i}^{\sigma_f} K(\sigma') d\sigma' &= H_0 \tau_0 \int_{\sigma_i}^{\sigma_f} e^{\sigma'} d\sigma' \\
&= H_0 \tau_0 [e^{\sigma_f} - e^{\sigma_i}] \\
&= H_0 [\tau_f - \tau_i]
\end{align}
Therefore:
\begin{equation}
U_\sigma(\sigma_f, \sigma_i) = \exp\left(-\frac{i}{\hbar} H_0 (\tau_f - \tau_i)\right) = U_\tau(\tau_f, \tau_i)
\end{equation}
\end{proof}

\begin{theorem}[Unitary Equivalence for Time-Dependent Hamiltonians]
\label{thm:unitary_equiv_time_dependent}
For general time-dependent Hamiltonians $H(\tau)$, the time-ordered evolution operators satisfy:
\begin{equation}
\mathcal{T} \exp\left(-\frac{i}{\hbar}\int_{\tau_i}^{\tau_f} H(\tau') d\tau'\right) = \mathcal{T} \exp\left(-\frac{i}{\hbar}\int_{\sigma_i}^{\sigma_f} K(\sigma') d\sigma'\right)
\end{equation}
where $\mathcal{T}$ denotes time-ordering.
\end{theorem}

\begin{proof}
The key insight is that the measure transformation preserves the integral:
\begin{align}
\int_{\tau_i}^{\tau_f} H(\tau') d\tau' &= \int_{\sigma_i}^{\sigma_f} H(\tau_0 e^{\sigma'}) \tau_0 e^{\sigma'} d\sigma' \\
&= \int_{\sigma_i}^{\sigma_f} K(\sigma') d\sigma'
\end{align}

For the time-ordered exponential, the ordering property is preserved because:
\begin{enumerate}
\item The transformation $\tau \mapsto \sigma$ is monotonic increasing
\item Time-ordering in $\tau$ corresponds to time-ordering in $\sigma$
\item The Dyson series expansion maintains the same structure in both coordinates
\end{enumerate}

The formal proof follows the standard analysis of time-ordered exponentials with the substitution $\tau = \tau_0 e^\sigma$.
\end{proof}

\emph{Heisenberg Picture Note}: The equivalence of evolution operators immediately implies that Heisenberg picture operator evolution is also preserved. For any observable $A$, the time evolution $A_H(\tau) = U^\dagger(\tau) A U(\tau)$ in proper time corresponds exactly to $A_H(\sigma) = U^\dagger(\sigma) A U(\sigma)$ in log-time, ensuring that all physical observables evolve identically in both coordinate systems.

\subsection{Asymptotic Silence}
\label{subsec:asymptotic_silence}

One of the most remarkable properties of the log-time formulation is the asymptotic silence of the effective generator as $\sigma \to -\infty$. This property provides a natural regularization of the quantum evolution near classical singularities.

\emph{Scope Note}: Asymptotic silence is a property of the evolution generator $K(\sigma)$ in log-time coordinates and provides computational advantages for quantum evolution near $\tau \to 0^+$. This is distinct from the geometric singularity treatment, which we address separately through Weyl transformations in Section~\ref{sec:cosmology}.

\begin{theorem}[Asymptotic Silence Property]
\label{thm:asymptotic_silence}
Let $H(\tau)$ be a Hamiltonian satisfying appropriate regularity conditions near $\tau = 0^+$. Then the effective generator $K(\sigma) = \tau_0 e^\sigma H(\tau_0 e^\sigma)$ exhibits asymptotic silence:
\begin{equation}
\lim_{\sigma \to -\infty} K(\sigma) = 0
\end{equation}
provided that $H(\tau)$ does not diverge faster than $\tau^{-1}$ as $\tau \to 0^+$.
\end{theorem}

\begin{proof}
As $\sigma \to -\infty$, we have $\tau = \tau_0 e^\sigma \to 0^+$. If $H(\tau)$ satisfies $\|H(\tau)\| \leq C \tau^{-\alpha}$ for some $\alpha < 1$ and constant $C$, then:
\begin{align}
\|K(\sigma)\| &= \tau_0 e^\sigma \|H(\tau_0 e^\sigma)\| \\
&\leq \tau_0 e^\sigma \cdot C (\tau_0 e^\sigma)^{-\alpha} \\
&= C \tau_0^{1-\alpha} e^{\sigma(1-\alpha)}
\end{align}
Since $1-\alpha > 0$, we have $\lim_{\sigma \to -\infty} e^{\sigma(1-\alpha)} = 0$, establishing asymptotic silence.
\end{proof}

\begin{corollary}[Finite Phase Accumulation]
Under the conditions of Theorem~\ref{thm:asymptotic_silence}, the total phase accumulated from $\sigma = -\infty$ to any finite $\sigma_f$ is finite:
\begin{equation}
\int_{-\infty}^{\sigma_f} \|K(\sigma')\| d\sigma' < \infty
\end{equation}
\end{corollary}

This finite phase accumulation is crucial for the mathematical well-posedness of quantum evolution with initial conditions specified in the asymptotic past.

\subsection{Observable Equivalence}
\label{subsec:observable_equivalence}

The physical content of quantum mechanics is contained in expectation values of observables. I demonstrate that these are preserved under the log-time transformation.

\begin{theorem}[Heisenberg Picture Equivalence]
\label{thm:heisenberg_equivalence}
For any observable $A$, the Heisenberg picture evolution in $\tau$ and $\sigma$ coordinates yields identical results:
\begin{equation}
\langle A \rangle_\tau(t) = \langle A \rangle_\sigma(\sigma)
\end{equation}
where $\sigma = \log(t/\tau_0)$ and the expectation values are computed with respect to quantum states evolved in the respective coordinate systems.
\end{theorem}

\begin{proof}
The Heisenberg picture observable evolves as:
\begin{align}
A_H^\tau(t) &= U_\tau^\dagger(t,0) A U_\tau(t,0) \\
A_H^\sigma(\sigma) &= U_\sigma^\dagger(\sigma,0) A U_\sigma(\sigma,0)
\end{align}
By the unitary equivalence established in Theorems~\ref{thm:unitary_equiv_constant} and~\ref{thm:unitary_equiv_time_dependent}, $U_\tau(t,0) = U_\sigma(\sigma,0)$ where $\sigma = \log(t/\tau_0)$. Therefore:
\begin{equation}
A_H^\tau(t) = A_H^\sigma(\sigma)
\end{equation}
and all expectation values are preserved.
\end{proof}

\subsection{Density Matrix Evolution}
\label{subsec:density_matrix}

For mixed quantum states described by density matrices, the preservation of physical predictions requires that density matrix evolution be equivalent in both coordinate systems.

\begin{theorem}[Density Matrix Equivalence]
\label{thm:density_matrix_equivalence}
The density matrix $\rho(\tau)$ evolving according to the Liouville-von Neumann equation in proper time:
\begin{equation}
i\hbar \frac{\partial \rho}{\partial \tau} = [H(\tau), \rho]
\end{equation}
is equivalent to the density matrix $\rho(\sigma)$ evolving in log-time:
\begin{equation}
i\hbar \frac{\partial \rho}{\partial \sigma} = [K(\sigma), \rho]
\label{eq:density_matrix_evolution}
\end{equation}
with $\rho(\tau) = \rho(\sigma)$ when $\sigma = \log(\tau/\tau_0)$.
\end{theorem}

\begin{proof}
The proof follows directly from the chain rule transformation and the relationship between $H(\tau)$ and $K(\sigma)$:
\begin{align}
i\hbar \frac{\partial \rho}{\partial \tau} &= i\hbar \frac{1}{\tau} \frac{\partial \rho}{\partial \sigma} \\
&= \frac{1}{\tau_0 e^\sigma} i\hbar \frac{\partial \rho}{\partial \sigma}
\end{align}
Setting this equal to $[H(\tau), \rho]$ and multiplying by $\tau_0 e^\sigma$:
\begin{equation}
i\hbar \frac{\partial \rho}{\partial \sigma} = \tau_0 e^\sigma [H(\tau_0 e^\sigma), \rho] = [K(\sigma), \rho]
\end{equation}
\end{proof}

\subsection{Non-Commuting Hamiltonians}
\label{subsec:noncommuting_hamiltonians}

A critical test of the framework is its behavior for non-commuting time-dependent Hamiltonians, where time-ordering becomes essential.

\begin{theorem}[Non-Commuting Hamiltonian Evolution]
\label{thm:noncommuting_evolution}
For Hamiltonians $H(\tau_1)$ and $H(\tau_2)$ that do not commute at different times, the time-ordered evolution preserves the non\-com\-mu\-ta\-tiv\-i\-ty structure in log-time coordinates:
\begin{equation}
[H(\tau_1), H(\tau_2)] \neq 0 \Rightarrow [K(\sigma_1), K(\sigma_2)] \neq 0
\end{equation}
where $\sigma_i = \log(\tau_i/\tau_0)$.
\end{theorem}

\begin{proof}
The commutator in log-time coordinates becomes:
\begin{align}
[K(\sigma_1), K(\sigma_2)] &= [\tau_0 e^{\sigma_1} H(\tau_0 e^{\sigma_1}), \tau_0 e^{\sigma_2} H(\tau_0 e^{\sigma_2})] \\
&= \tau_0^2 e^{\sigma_1} e^{\sigma_2} [H(\tau_1), H(\tau_2)]
\end{align}
Since $\tau_0^2 e^{\sigma_1} e^{\sigma_2} > 0$, the commutator structure is preserved with a positive scaling factor.
\end{proof}

\subsection{Computational Validation of Quantum Evolution}
\label{subsec:computational_validation_quantum}

I have implemented comprehensive numerical validation of all quantum mechanical properties. The validation suite confirms:

\begin{enumerate}
\item \textbf{Evolution operator equivalence}: For both constant and time-dependent Hamiltonians, $\|U_\tau - U_\sigma\| < 10^{-10}$
\item \textbf{Unitary preservation}: $\|U^\dagger U - I\| < 10^{-12}$ in both coordinate systems
\item \textbf{Observable expectation values}: Agreement within $10^{-10}$ tolerance for all test observables
\item \textbf{Density matrix evolution}: Trace preservation and positivity maintained to machine precision
\item \textbf{Asymptotic silence}: Verified convergence of $K(\sigma) \to 0$ as $\sigma \to -\infty$ for test Hamiltonians
\end{enumerate}

\subsection{Physical Interpretation and Implications}

The quantum mechanical results I have established demonstrate that:

\begin{itemize}
\item \textbf{Complete Equivalence}: The log-time reparameterization preserves all quantum mechanical predictions while providing new mathematical structure for analysis.

\item \textbf{Regularized Evolution}: Asymptotic silence provides a natural regularization mechanism for quantum evolution near classical singularities.

\item \textbf{Additive Structure}: The effective generator $K(\sigma)$ inherits the additive properties of log-time, making it compatible with quantum mechanical phase evolution.

\item \textbf{Operational Consequences}: Different sampling strategies in $\tau$ versus $\sigma$ coordinates can lead to measurably different experimental protocols, providing potential observational signatures.
\end{itemize}

\subsection{The Problem of Time in Deparameterization}
\label{subsec:problem_of_time}

The LTQG framework's relationship to canonical quantum gravity reveals a fundamental conceptual limitation regarding the Problem of Time.

\begin{definition}[The Problem of Time]
In canonical quantum gravity, diffeomorphism invariance leads to the Wheeler-DeWitt constraint:
\begin{equation}
\hat{H} \psi = 0
\end{equation}
This implies $\partial \psi/\partial t = 0$, creating the ``frozen formalism'' where quantum states appear static, conflicting with physical evolution.
\end{definition}

\paragraph{LTQG's Deparameterization Approach}
The framework sidesteps this issue through deparameterization:
\begin{enumerate}
\item \textbf{Minisuperspace Context}: The analysis is restricted to cosmological minisuperspace where a global time function $\tau$ (from scalar field $\phi$) can be chosen.
\item \textbf{Gauge Choice}: The scalar field $\tau$ serves as internal time through gauge choice: $\tau = \tau(\phi)$, not fundamental resolution.
\item \textbf{Reduced Hamiltonian}: $K(\sigma) = \tau_0 e^\sigma H(\tau_0 e^\sigma)$ becomes the evolution generator after solving the constraint $\hat{H} = 0$ in this gauge.
\end{enumerate}

\begin{proposition}[Logical Inconsistency in Full Theory]
\label{prop:logical_inconsistency}
LTQG cannot simultaneously preserve ``complete physical content of GR'' (including diffeomorphism invariance) and maintain well-defined quantum evolution.
\end{proposition}

\begin{proof}
If diffeomorphism invariance is preserved, then $\hat{H} \psi = 0$ must hold, implying:
\begin{equation}
\frac{\partial \psi}{\partial t} = 0
\end{equation}
But LTQG claims well-defined evolution:
\begin{equation}
i\hbar \frac{\partial \psi}{\partial \sigma} = K(\sigma) \psi \neq 0
\end{equation}
Since $\sigma = \log(\tau/\tau_0)$ with $\tau$ related to coordinate time, these requirements are logically inconsistent unless diffeomorphism invariance is broken.
\end{proof}

\paragraph{Limitation Assessment}
\begin{itemize}
\item \textbf{Minisuperspace Success}: The deparameterization works effectively in cosmological minisuperspace where global time functions exist.
\item \textbf{Full Theory Challenge}: Extension to full field theory faces the inability to choose global time functions uniquely due to diffeomorphism invariance.
\item \textbf{Gauge vs. Fundamental}: The ``solution'' represents a gauge choice rather than fundamental resolution of the constraint algebra.
\item \textbf{Physical Interpretation}: The framework provides computational tools for minisuperspace quantum cosmology but cannot claim to solve the full Problem of Time.
\end{itemize}

This analysis establishes that LTQG is a powerful deparameterization technique for minisuperspace applications rather than a fundamental solution to canonical quantum gravity's temporal difficulties.

The mathematical rigor and computational validation ensure that these quantum mechanical foundations provide a solid basis for the cosmological and field theory applications that follow in subsequent sections.