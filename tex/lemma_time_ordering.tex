\documentclass[11pt]{article}
\usepackage{amsmath,amssymb,amsthm,mathtools,bm}
\usepackage[margin=1in]{geometry}
\usepackage{hyperref}
\hypersetup{colorlinks=true, linkcolor=blue, citecolor=blue, urlcolor=blue}
\title{Time-Ordering Invariance Under Monotone Clock Changes}
\author{Addendum to LTQG: $\sigma=\log(\tau/\tau_0)$}
\date{\today}

\newtheorem{theorem}{Theorem}
\newtheorem{lemma}{Lemma}
\theoremstyle{definition}
\newtheorem{definition}{Definition}
\newtheorem{assumption}{Assumption}
\newtheorem{remark}{Remark}

\begin{document}
\maketitle

\section*{Context}
Let $\tau\in I\subset\mathbb{R}$ with $I$ an interval and let $\sigma=F(\tau)$ be a \emph{monotone} $C^1$ change of clock with $F':I\to(0,\infty)$.
Consider a (possibly time-dependent) Schr\"odinger generator $H(\tau)$ on a Hilbert space $\mathcal{H}$.
Write $U_\tau(t_2,t_1)$ for the $\tau$-time propagator solving
\begin{equation}
 i\,\partial_\tau U_\tau(\tau,\tau_0)=H(\tau)\,U_\tau(\tau,\tau_0),\qquad U_\tau(\tau_0,\tau_0)=\mathbf{1},
\end{equation}
and define the $\sigma$-time generator by the usual chain rule
\begin{equation}
 \tilde H(\sigma)\coloneqq \frac{d\tau}{d\sigma} H(\tau(\sigma)) = \frac{H(\tau(\sigma))}{F'(\tau(\sigma))}.
\end{equation}
Denote by $U_\sigma(\sigma_2,\sigma_1)$ the corresponding $\sigma$-time propagator.

\section*{Main Lemma (Time-Ordering Invariance)}
\begin{lemma}[Chronological invariance under monotone clocks]\label{lem:order}
If $F$ is strictly increasing, then for any $\tau_1\le \tau_2$ we have
\begin{equation}
 \mathcal{T}_\tau\!\left[\exp\!\Big(-i\!\int_{\tau_1}^{\tau_2}\! H(\tau)\,d\tau\Big)\right]
 \;=\;
 \mathcal{T}_\sigma\!\left[\exp\!\Big(-i\!\int_{\sigma_1}^{\sigma_2}\! \tilde H(\sigma)\,d\sigma\Big)\right],
\end{equation}
with $\sigma_j=F(\tau_j)$. In particular, $U_\tau(\tau_2,\tau_1)=U_\sigma(\sigma_2,\sigma_1)$.
\end{lemma}

\begin{proof}
Strict monotonicity implies that the map $\tau\mapsto\sigma=F(\tau)$ preserves order: $\tau_a<\tau_b\iff\sigma_a<\sigma_b$.
Hence the chronological partitions used to define Dyson expansions can be transported bijectively.
Start from the Dyson series in $\tau$:
\begin{align}
U_\tau(\tau_2,\tau_1) &= \mathbf{1} + \sum_{n\ge1} (-i)^n \!\!\!\!\!\!
\int_{\tau_1\le t_n\le\cdots\le t_1\le \tau_2} \!\!\!\!\!\!\!\! H(t_1)\cdots H(t_n)\, dt_1\cdots dt_n .
\end{align}
Change variables $s_k=F(t_k)$; then $dt_k = \frac{dt_k}{ds_k}\,ds_k=\frac{1}{F'(t_k)}\,ds_k$ and the ordered simplex is mapped to
$\sigma_1\le s_n\le\cdots\le s_1\le \sigma_2$ by monotonicity.
Each factor transforms as $H(t_k)\,dt_k = \tilde H(s_k)\,ds_k$ by definition of $\tilde H$.
Therefore every $n$-simplex integral equals the corresponding $\sigma$-time Dyson term, yielding the identity.
\end{proof}

\section*{Existence \& Uniqueness Hypotheses}
\begin{assumption}[Bounded case]
$H(\tau)$ is strongly measurable and $\sup_{\tau\in I}\|H(\tau)\|<\infty$. Then the Dyson series converges in operator norm and defines a unique unitary propagator.
\end{assumption}

\begin{assumption}[Kato Class (unbounded case)]
$H(\tau)$ is self-adjoint on a dense domain $D$ (independent of $\tau$), $\tau\mapsto H(\tau)\psi$ is continuous $\forall\psi\in D$, and the family is stable in the sense of Kato (domain invariance and suitable bounds).
Under these hypotheses the evolution family $U_\tau$ exists and is unique; the reparameterization $\tilde H(\sigma)=\frac{d\tau}{d\sigma}H(\tau(\sigma))$ preserves the same class, so $U_\sigma$ exists and equals $U_\tau$ by Lemma~\ref{lem:order}.
\end{assumption}

\begin{theorem}[Unitary equivalence under monotone reparameterization]
Under either set of hypotheses above, $U_\tau(\tau_2,\tau_1)=U_\sigma(\sigma_2,\sigma_1)$ for $\sigma_j=F(\tau_j)$. Consequently, spectra of Heisenberg-evolved observables and transition probabilities coincide for $\tau$- and $\sigma$-descriptions.
\end{theorem}

\begin{remark}[Application to LTQG]
For $F(\tau)=\log(\tau/\tau_0)$ with $\tau>0$ we have $\frac{d\tau}{d\sigma}=\tau$ and hence $\tilde H(\sigma)=\tau(\sigma)\,H(\tau(\sigma))$.
The lemma shows that the reparameterization leaves all physical predictions invariant while making multiplicative early-time structure additive in $\sigma$.
\end{remark}

\end{document}