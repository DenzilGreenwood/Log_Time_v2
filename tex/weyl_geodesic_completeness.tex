\documentclass[11pt]{article}
\usepackage{amsmath,amssymb,amsthm,mathtools,bm}
\usepackage[margin=1in]{geometry}
\usepackage{hyperref}
\title{Geodesic Checks in the Weyl Frame for Log-Time Reparameterization}
\author{Addendum to LTQG}
\date{\today}
\newtheorem{prop}{Proposition}
\newtheorem{remark}{Remark}
\begin{document}
\maketitle

\section*{Setup}
Consider a spatially flat FLRW metric in proper time $\tau>0$:
\begin{equation}
 ds^2 = -d\tau^2 + a(\tau)^2 d{x}^2,\quad a(\tau)=\left(\frac{\tau}{\tau_*}\right)^p,\; p>0.
\end{equation}
Introduce the log-time $\sigma=\log(\tau/\tau_0)$ so that $\tau=\tau_0 e^\sigma$.
Define the Weyl-rescaled \emph{log-time frame} metric $\tilde g$ with conformal factor $\Omega(\tau)=\tau^{-1}$:
\begin{equation}
 \tilde g_{\mu\nu}=\Omega^2 g_{\mu\nu}=\tau^{-2} g_{\mu\nu}\,,
 \qquad d\tilde s^2 = -\tau^{-2} d\tau^2 + \tau^{-2} a(\tau)^2 d{x}^2 .
\end{equation}
In $\sigma$-coordinates this is
\begin{equation}
 d\tilde s^2 = -d\sigma^2 + \left(\frac{\tau_0}{\tau_*}\right)^{2p} e^{2(p-1)\sigma}\, d{x}^2 .
\end{equation}

\section*{Null geodesics}
Null geodesics are conformally invariant as curves (their images are preserved) though the affine parameter rescales.
For radial null curves ($d{x}=dr$),
\begin{equation}
 0 = -d\sigma^2 + C^2 e^{2(p-1)\sigma} dr^2,\qquad C\coloneqq \left(\frac{\tau_0}{\tau_*}\right)^p .
\end{equation}
Hence $dr/d\sigma = \pm C^{-1} e^{-(p-1)\sigma}$ and the coordinate distance from $\sigma=-\infty$ to a finite $\sigma_1$ is
\begin{equation}
 \Delta r = \int_{-\infty}^{\sigma_1} \!\! C^{-1} e^{-(p-1)\sigma}\, d\sigma = 
 \begin{cases}
  \infty, & p\le 1,\\[4pt]
  \dfrac{1}{C(p-1)} e^{-(p-1)\sigma_1}, & p>1.
 \end{cases}
\end{equation}
Thus for radiation ($p=\tfrac12$) and matter ($p=\tfrac23$), the comoving distance to the past boundary $\sigma\to-\infty$ diverges in the Weyl frame.

\section*{Affine parameter near $\sigma\to -\infty$}
Let $\lambda$ be an affine parameter for the null geodesic in the Weyl frame.
For a metric of the form $d\tilde s^2 = -d\sigma^2 + B(\sigma)^2 dr^2$, radial nulls satisfy $d r/d\sigma = \pm 1/B(\sigma)$ and one finds
\begin{equation}
 \frac{d^2\sigma}{d\lambda^2} + \frac{B'(\sigma)}{B(\sigma)} \left(\frac{d\sigma}{d\lambda}\right)^2 = 0
 \quad\Rightarrow\quad
 \frac{d\sigma}{d\lambda} \propto B(\sigma)^{-1}.
\end{equation}
With $B(\sigma)=C\,e^{(p-1)\sigma}$ we obtain $d\lambda \propto e^{-(p-1)\sigma} d\sigma$; hence
\begin{equation}
 \int_{-\infty}^{\sigma_1} d\lambda \propto \int_{-\infty}^{\sigma_1} e^{-(p-1)\sigma} d\sigma =
 \begin{cases}
  \infty, & p\le 1,\\[4pt]
  \dfrac{1}{p-1} e^{-(p-1)\sigma_1}, & p>1.
 \end{cases}
\end{equation}
\textbf{Conclusion:} for $p\le 1$ (including radiation and matter), past-directed null geodesics in the Weyl frame have \emph{infinite} affine length to $\sigma\to -\infty$; i.e., the past boundary is null-geodesically complete in this frame.

\section*{Timelike geodesics (comoving observers)}
Consider comoving worldlines ($r=\text{const}$). In the Weyl frame their proper time is simply $\tilde{\tau}=\sigma$ up to an additive constant, because $d\tilde s^2=-d\sigma^2$.
Therefore the proper time to the past boundary is
\begin{equation}
 \Delta \tilde{\tau} = \int_{-\infty}^{\sigma_1} d\sigma = \infty.
\end{equation}
Thus comoving timelike geodesics are also complete to the past in the Weyl frame.

\begin{prop}[Geodesic completeness criteria in the Weyl frame]
For spatially flat FLRW with $a(\tau)\propto \tau^p$ and conformal rescaling $\tilde g=\tau^{-2} g$:
\begin{itemize}
 \item Past null geodesics are complete for $p\le 1$.
 \item Comoving timelike geodesics are complete for all $p>0$.
\end{itemize}
\end{prop}

\begin{remark}[Interpretation for LTQG]
The $\sigma$-time boundary ($\sigma\to -\infty$) corresponds to $\tau\to 0^+$.
In the Weyl frame, standard early-time cosmologies ($p\le 1$) become null- and timelike-complete towards this boundary.
This matches the ``asymptotic silence'' narrative: dynamics \emph{slow} in $\sigma$ and geodesic paths accumulate infinite parameter length before reaching $\sigma=-\infty$.
\end{remark}

\end{document}